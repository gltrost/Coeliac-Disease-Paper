\documentclass{article}
\usepackage{xcolor}


\usepackage{INTERSPEECH2021}
\usepackage{libertine}
\usepackage[a4paper,left=2cm,right=2cm,top=3.0cm,bottom=2.5cm]{geometry} 
\usepackage[textwidth=20mm,colorinlistoftodos,textsize=tiny]{todonotes} %for \todo{}
\usepackage{soul}

%for graph
\usepackage{tikz}



\title{The Effects on the Brain of Immune-Response from Coeliac Disease}
\name{\textbf{Greg Trost},  \\ under the supervision of Dr. Russell Morton, University of New Mexico}
%The maximum number of authors in the author list is twenty. If the number of contributing authors is more than twenty, they should be listed in a footnote or in acknowledgement section, as appropriate.
\email{gltrost2@gmail.com}
\address{}

\begin{document}
\setlength{\parskip}{0pt}

\maketitle
% 
\begin{abstract}
  A correlation between Coeliac Disease (CD) and neurological comorbidities has been studied and recorded in various surveys \cite{Bushara,Cicarelli}, however, the possible causes of these neurological  conditionshave not been well studied. The links between CD and neurological issues could be due to one or a combination of many different mechanisms: underlying genes that code for both CD and neurological issues, effects of neurochemicals like serotonin and (nor-) epinephrine found in the gut, hormone levels in the the gut microbiota \cite{Obrenovich2}, and the role of neuropeptides which communicate  between the gut and brain are a few of the possible mechanisms that could be explored. In this paper, we specifically explore the role of CD's over-activated immune-response and its effects on the brain.
  
  Looking at the available literature, we listed several known neurological issues that are correlated with CD, and then looked to see if these issues were at least in part caused by a weakened Blood-Brain Barrier (BBB). \textcolor{red}{you need to connect the weakened BBB to your CD mechanism theories}Our research found that the immune-response of CD may play a small role in these neurological issues, but that more research needs to be done to make strong conclusions.
  \textcolor{blue}{EXPLAIN THE CONNECTION FROM IMMUNE RESPONSE TO INFLAMMATION TO BBB TO ISSUES}
  
  \section{Weakened BBB implies issues (give examples)}
  \textcolor{blue}{Give examples of where weakened BBB causes issues brain}
  
\end{abstract}
\noindent\textbf{Index Terms}: Coeliac Disease, Blood-Brian Barrier, Inflammation

\section{Introduction}
  Coeliac Disease (CD) is an autoimmune disorder that affects about 1$\%$ of most populations \cite{Lebwohl}. The disease is caused in susceptible individuals by the ingestion of gluten whereby gliadin, a component of gluten, makes its way into the lamina propria of the intestine. \textcolor{red}{The presence of gliadin in the intestines of these individuals}  triggers an auto-immune response damaging the lining of the intestines and leading to villous atrophy. Symptoms of CD can range ``... from severe malabsorption to minimally symptomatic or non-symptomatic presentations''\cite{Lebwohl}.
  Looking past the primary symptoms of CD, studies have shown correlations between CD and neurological issues including ataxia, migraine/headaches, peripheral neuropathy, epilepsy and depression. The reasons for these correlations are still unknown.

  In this paper, we explore the hypothesis that the immune response of CD may contribute to some of these neurological issues. We mainly, but not entirely, focus on the theory that immune-mediated inflammation in CD may cause neuroinflammation and a weakened Blood-Brain Barrier (BBB), and that this weakened BBB may be related to the above mentioned neurological issues. The hypothesis we present is that the immune-response triggered in CD does contribute to many of these issues.
  Section 2 summarizes the pathology of CD and CD-mediated inflammation. Section 3 summarizes which neurological issues are shown to be correlated with CD and how they may arise due to CD's immune-response. Section 4 shows drawbacks and missing information needed to better support our hypothesis.

\section{Coeliac Disease, Immune-Response and Inflammation}
One way to explore the theory that CD immune-responses cause the CD-correlated neurological issues is to see if the neurological issues in question are caused by immune responses. 

\includegraphics[width=\textwidth / 3 ]{LaTeX/logic.png}

The goal is to assess the strength of the black lines in the above diagrams with little-to-no knowledge of the red lines that may also be at play. Since the correlations between CD and the other neurological issues are far from bijective, showing the causation of CD to these issues must be approached with particular nuance and care.

CD is an autoimmune disorder caused by gluten consumption which causes extra-intestinal inflammation. Specifically, ``nonself antigens, including gluten, \textcolor{red}{should gliadin be mentioned here?}other food antigens and microorganism components gain access into the lamina propria and activate inflammatory cells to release cytokines (including IL-15) that cause innate immune inflammation" \cite{Leffler DA}. Then ``activated inflammatory cells migrate to the intestinal epithelial cells... responsible for extraintestinal  symptoms \cite{Reilly}"\cite{Leffler DA}. \textcolor{red}{(what are some extraintestinal symptoms?)} Some common extra-intestinal issues resulting from CD include diarrhrea, malabsobrtion (notably iron and B$_{12}$ deficiencies) and anemia (caused by both malabsorption and inflammation) \cite{Leffler DA}. 

Despite the clear role inflammation plays in CD, there isn't much research as to whether CD-inflammation leads to significant neuro-inflammation, or if neurological issues associated with CD are explained by inflammation issues. One paper on \textit{non-Coeliac} gluten sensitivity purports a direct connection between the reaction to gluten and neuroinflammation: ``Systemic administration of proinflammatory cytokines TNF-$\alpha$ or IL-1$\beta$ increased IL1-$\beta$, TNF-$\alpha$ and IL-6 mRNAs, but not the anti-inflammatory IL-10 cytokine, in the NTS (nucleus tractus solitarius) [193]. Recent studies have documented that \st{the
} microvasculature inflammation may occur in the brainstem, specifically in the NTS"\cite{Daulatzai}. The authors further conclude a direct connection: ``GI inflammation $\rightarrow$ systemic inflammation $\rightarrow$ neuroinflammation $\rightarrow$ cognitive impairment." That said, non-Coeliac gluten sensitivity doesn't have exactly the same pathology as CD. For example IL1-$\beta$ and TNF-$\alpha$ don't appear any more often in CD-patients than in controls \cite{Bjork}. It can at least be hypothesized though that the pathology of non-Coeliac Gluten Sensitivity is similar enough to CD that similar effects on the brain could arise. 

\subsection{Antibodies Present in Coeliac Disease}

The types of antibodies that are produced during the autoimmune response of CD are well-recorded. Via the HLA-DQ8 and HLA-DQ2 haplotypes, IFN-$\gamma$, IL-12p70 \cite{Bjork,DePaolo}, and IL-15 \cite{DePaolo} \cite{Yokoyama} \cite{Abadie} are produced in the presence of gluten in the intestines. Studies have even gone as far as showing that suppressing IL-15 can alleviate/prevent villous atrophy caused by CD in mice \cite{DePaolo} \cite{Yokoyama}. On the other hand, no elevation in common cytokines TNF-$\alpha$, IL-1$\beta$, IL-2, IL-4, IL-8 and IL-10 are found in CD \cite{Bjork,Aflatoonian}. 

\subsection{Leaky Gut Syndrome}

One of the major new theories in the gut-brain access, and a theory that has a central role to this paper is \textit{Leaky Gut Syndrome} (LGS). Put succintly, LGS is ``...an increase in the permeability of the intestinal mucosa, which could allow bacteria, toxic digestive metabolites, bacterial toxins, and small molecules to ‘leak’ into the bloodstream" \cite{Obrenovich}. CD is indeed one such autoimmune disease that damages the mucosa of the intestine, leading to such "leakage" of toxins into the blood stream. 

Obrenovich et al. include the BBB as an area affected by LGS, stating ``The gut-brain axis also has a ‘gut-immunity axis’, beyond innate immunity or complement and phagocytic expression, such as the creation and maintenance of the blood brain barrier \cite{Braniste}"\cite{Obrenovich2}, and where "indirect effects of the gut microbiota or direct effects of antibiotics or probiotics on the innate immune system can result in alterations in the circulating levels of proinflammatory and anti-inflammatory cytokines to modify brain function"\cite{Obrenovich2}. 

Studies show that changes to the gut-microbiota can cause weakenings to the BBB. Braniste et al. showed in one study comparing germ-free mice to pathogen-free mice that germ-free mice had much more permeable BBB: "germ-free mice, beginning with intrauterine life, displayed increased BBB permeability compared to pathogen-free mice with a normal gut flora"\cite{Braniste2}.

\includegraphics[width=\textwidth / 3 ]{LaTeX/gut_BBB.png}

\textit{Fig. 4I-J from Braniste et al. showing pathogen-free mouse (PF) BBB tight junctions (left image and black-bar) vs. germ-free (GF) BBB tight junctions (right image and white bar). The white lines show major junction-weakening in the GF-image.}

Overall, LGS may work as a good model to think of CD and neurological issues. Studying the connection between CD immune-response and neurological issues could neatly add to the general knowledge-base of LGS.

\section{Neurological Issues Correlated with CD and their relationship to Neuroinflammation}

\subsection{Cerebral Ataxias}

Studies show a correlation between Cerebral Ataxias (CAs) and CD, with Bushara et al. citing four different studies showing a correlation: ``In patients with cerebellar ataxia of unknown cause, biopsy examination–proven CD was found in 16$\%$ (4 of 25)\cite{Hadjivassiliou}, 16.7$\%$ (4 of 24)\cite{Luostarinen}, 12.5$\%$ (3 of 24)\cite{Pellecchia}, and 1.9$\%$ (2 of 104)\cite{Burk}" \cite{Bushara}. All of these studies show CD at rates higher than most population-percentages CD diagnosis, which sit around 1$\%$\cite{Lebwohl}. 

Neuroinflammation is also well cited as playing a role in CAs. As shown in \cite{Duarte}, ``human  brain  samples  were  used  to  confirm  BBB  permeability  by evaluating  fibrinogen (a protein found in the blood's plasma)  extravasation,  co-localization  of  ataxin-3  aggregates with brain blood vessels and neuroinflammation" and ``Given the essential role of BBB in maintaining homeostasis, its dysfunction possibly exacerbates MJD (Machado-Joseph disease)-associated neuropathological events, such as neuroinflammation" where MJD is a form of Ataxia, specifically spinocerebellar ataxia-3. Thus the correlation between CAs and CD, at least with respect to MJD, is decently strong.

There is also some support for the claim that neuroinflammation plays a causal role in CA. A recent study from Hong et al. in \cite{Hong} showed in that mice with lipoplysaccharide-induced CA resulted in neuroinflammation which lead to in the "...increased expression of the proinflammatory molecules IL-1$\beta$, TNF-$\alpha$, MCP-1, and MIP-1$\alpha$". The study supports the link between CAs and neuroinflammation concluding "A recent accumulation of clinical and preclinical evidence supports the hypothesis that cerebellar inflammation plays a significant role in the progression of CAs". However, there isn't much more evidence showing a causal link from neuroinflammation to CAs in this study, only evidence in the reverse direction.

Thus at the moment there is little-to-no evidence that neuroinflammation causes CAs. Evidence points to the notion that neuroinflammation is correlated Ataxia to some degree, though more likely than not CAs induce the inflammation, rather than the other way around. 

\subsection{Epilepsy}

Epilepsy has been shown to be more prevalent in CD patients (and vice versa) compared to the general population \cite{Bushara}. It's also been suggested that CD is linked more specifically to seizures that present cerebral calcification, but this hypothesis is not well supported \cite{Bushara}. 

Epilepsy is known to be correlated with a weakened BBB, but it's important to understand the causality between Epilepsy and a weakened BBB to know if CD may play a role. Löscher et al.'s paper, aptly named \textit{Functional Alterations of the Blood-Brain Barrier during Epileptogenesis and Epilepsy: A Cause, Consequence, or Both?}, concluded \textit{both}: Namely epileptogenic brain injuries may lead to a weakened BBB, which in turn leads to an uptake in various cytokines, which leads to a decreased threshold for spreading depolarization, finally leading back to an increased risk of seizure\cite{Loscher}. 

\includegraphics[width=\textwidth / 2 ]{LaTeX/Epilepsy_bbb.jpeg}

\textit{Diagram from Löscher et al highlighting the feedback-loop of epilepsy and BBB-dysfunction}

Thus it's reasonable to suggest that a weakened BBB from CD's immune-mediated inflammation plays a role in epilepsy. Though Löscher et al.'s study specifically pointed to TGF-$\beta$ as the main culprit to inflammation-induced seizure, there's room for hypothesis that other CD-specific cytokines may play the same role as TGF-$\beta$.  The evidence for a correlation between a weakened BBB and epilepsy is strong \cite{Loscher} \cite{Janigro}, and causation of BBB and epilepsy is reasonably strong, though not conclusive.

\subsection{Migraines/Headaches}

Migraine and Headache disorders are shown to be correlated with CD in a few studies mainly displayed in \cite{Cicarelli}, where 46$\%$ of CD patients reported experiencing Migraines vs 29$\%$ for non-CD controls. 

The pathology of Migraines/Headaches isn't well studied, and could be due to multiple issues including estrogen-levels \cite{Chai} and "cortical spreading depression-like events" \cite{Dodick}. However, the connection between Migraines/Headaches and BBB permeability is strong. A meta-analysis by DosSantos et al. found ``there is mounting evidence that BBB/BSCB/BNB disruptions participate in the complex mechanisms that initiate or maintain inflammatory, neuropathic pain, and migraine"\cite{DosSantos}. 

Another study by Mahmoudi et al. supports the idea that neuroinflammation was a cause of migraines by testing the effects of nitroglycerin (NTG) and cerebrolysin (CBL) on rats. After injecting NTG into rats, found an increase migraine, TNF-$\alpha$ and IL-1$\beta$ \cite{Mahmoudi} \textcolor{red}{this previous sentance does not make sense. Please re-word it.}. They concluded ``...it seems that CBL alleviates migraine-associated symptoms through reduction of the release of CGRP and PACAP, and inhibition of neurogenic inflammation by reducing TNF-$\alpha$ and IL-1$\beta$"\cite{Mahmoudi}. Authors He et al. found inflammatory links to chronic migraines by testing the effects of NTG on rats (similar to Mahmoudi et al.) and finding an uptake in NLRP3 and IL-1$\beta$ \cite{He}.

A slightly older study from 2001 meta-analysis by Kemper et al. found that TNF-$\alpha$ can induce migraines, and likewise TNF-$\alpha$ antibodies are shown to reduce migraine attacks \cite{Kemper}. Though CD isn't known to have any uptake in TNF-$\alpha$ specifically, this definitely supports the theory that immune-response contributes to migraines/headaches. Finally, Edvinsson et al. found ``The idea of neurogenic neuroinflammation in the trigeminal ganglion could explain the findings of inflammatory markers in patients with migraine"\cite{Edvinsson}. 

We rule the neuroinflammation-to-Migraines/Headaches hypothesis as probable. That said, migraines/headaches are far from being entirely understood, so due to the current lack of general knowledge of migraines/headaches, these findings should not be considered conclusive. 

\subsection{Depression}

Depression and CD are shown to be correlated in several studies. A survey from Cicarelli et al. showed patients with CD had a history of dysthymia (long lasting depression) at around 15$\%$ (n=176) vs 4$\%$ for for the non-CD patients (n=52) \cite{Cicarelli}. Another study \st{like one} by Ciacci et al. also found higher rates of self-reported depression in CD patients compared to control groups \cite{Ciacci}. One hypothesis for this correlation, found by Bushara et al., is that malabsorption and nutritional deficiencies caused by CD are what result in depression, specifically $B_6$ malabsorption as one theory for the correlation\cite{Bushara}.  

The link between depression and neuroinflammation is heavily studied and thought to be bidirectional. A study by Kohler et al. put it directly finding ``Accumulating evidence supports an association between depression and inflammatory processes, a connection that seems to be bidirectional... In addition, somatic inflammatory diseases and increased pro-inflammatory markers increase the risk for subsequent development of depression"\cite{Kohler}. Kohler et al. list studies showing how both inflammation and inflammatory agents can lead to depression. This conclusion is mainly based on the authors' findings that "pro-inflammatory treatment frequently results in psychiatric side effects. Up to 80$\%$ of (pro-inflammatory) IFN-$\alpha$ treated patients have been reported to suffer from mild to moderate depressive symptoms \cite{Friebe,Eggermont,Reichenberg}." 

One major finding is listed Kohler et al. listed that is especially interesting is a Danish nationwide study that found those who were hospitalized for autoimmune diseases were 45$\%$ more likely to have mood disorders, and  those who were hospitalized for for infection were 62$\%$ more likely to have mood disorders \textcolor{red}{(so you mean infection is more linked to mood disorders than autoimmune disease?)} \cite{Benros}. This survey's findings could be due to underlying correlations not listed (like an underlying causation that may induce both autoimmune disorders and mood disorders), however the survey is at least very promising for a causative role of immune-response to depression. 

Some studies show an even more connected relation between the gut and depression. A more direct review titled \textit{Inflammation: depression fans the flames and feasts on the heat} brings an even closer look at the connection between the gut and brain, claiming "Alterations in the gut microbiota shape physiology through contributions to inflammation, obesity, and mood, among other things"\cite{Kiecold-Glaser}. This study not only supports the likelihood of depression being caused in part by inflammation in general, but by inflammation from changes in the gut specifically. Plenty of evidence shows the reshaping of the gut microbiome from the immune response from CD, as laid out in our section on Leaky Gut Syndrome. 

Thus we find the connection between the CD immune response and depression as very probable. 

% One large review on neuroinflammation and depression by Troubat et.al states "...clarifying whether neuroinflammation is a consequence of MDD (Major Depression Disorder) or a key aetiological factor causing depression appears to be a chicken‐or‐egg dilemma"\cite{Troubat} though the study at least suggests that neuroinflammation plays a major part in understanding Depression. Likewise, a study on links between the Depression and neuroinflammation by Kim et.al. concluded "...despite the evidence that a dysfunctional immune and endocrine system contributes to the pathophysiology of depression, much research remains to be undertaken to clarify the cause and effect relationship between depression and neuroinflammation."\cite{Kim} Thus it would probably me premature to conclude neuroinflammation causes depression, though it may very well play into depression; a feed-back loop between neuroinflammation and depression. 


\subsection{Peripheral Neuropathy}

Peripheral Neuropathy (PN) has also been correlated with CD. Specifically Cicarelli et al. found that CD patients were much more likely to exhibit symptoms of PN than controls \cite{Cicarelli}. However, the survey done (1) lists cramps as a symptom of PN, while cramps may be caused by CD for a number of other gastrointestinal reasons \cite{Hourigan}, and (2) the survey doesn't list the overlap of reported symptom. Nevertheless the numbers were high enough and different enough from the control group to be worth speculating a correlation between PN and CD. 

PN plays a unique role in this paper, since it's the only issue we look at that does not have any known connection to neuroinflammation that we could find. Instead, Bushara et al. suggests looking at anti-ganglioside antibodies, which are found in autoimmune disorders and also play an important role in PN \cite{Bushara}. Furthermore, Volta et al. and Volta et al. found strong correlations between Anti-ganglioside antibodies and CD in surveys \cite{Volta,Volta2}. 

On the contrary, Briani et al. found that ``anti-ganglioside antibodies do not seem to correlate with gluten ingestion or with neurological manifestations in children with coeliac disease" \cite{Briani}. It's possible that anti-ganglioside antibodies increase in presence in adulthood, and PN is certainly more prevalent in aging populations \cite{Barrell}, but more research needs to be done before making this conclusion. 

\section{Strength of Evidence and Conclusions}

The idea that the immune response from CD is a player in the mentioned neurological issues is probable, but far from conclusive. Studies show that neuroinflammation plays a part in CD-correlated neurological issues like CAs, Migraines/Headaches and Depression, and even some still suggest a causation from neuroinflammation to these issues. Studies between immune-response and CAs are less conclusive.  As for PN, the evidence seems tipped in favor of there being a causation of CD-mediated proteins contributing to PN, but more research is still needed.

There is little to no direct evidence of cytokines specific to CD showing up in the correlated neurological issues. The most commonly observed immune agents in the neurological were TNF-$\alpha$, IL-1$\beta$ and TGF-$\beta$, among others. This doesn't at all rule out the idea that CD cytokines can mimic the roles seen by these other more common cytokines. One area of research and study would be to see how the agents specific to CD can replicate the  agents specific to the immune-responses seen in some of the studies and analyses shown here.

Overall, we speculate that CD plays into these neurological issues via a combination of different ways all playing a role in CD: serotonin levels, vitamin malabsorption, immune-response, and other issues found in LGS-like microbiome-diversity. Thus, given the complexity of the chemical, biological and neuronal systems involved and the early stages of research on CD, it is not surprising that these studies do not reach conclusive findings of CD's effects on the brain. 

\bibliographystyle{IEEEtran}


\begin{thebibliography}{9}
 \bibitem[1]{Leffler DA}
Leffler DA, Green PH, Fasano A. Extraintestinal manifestations of coeliac disease. Nat Rev Gastroenterol Hepatol. 2015 Oct;12(10):561-71. doi: 10.1038/nrgastro.2015.131. Epub 2015 Aug 11. PMID: 26260366.

\bibitem[2]{Daulatzai}
Daulatzai MA. Non-celiac gluten sensitivity triggers gut dysbiosis, neuroinflammation, gut-brain axis dysfunction, and vulnerability for dementia. CNS Neurol Disord Drug Targets. 2015;14(1):110-31. doi: 10.2174/1871527314666150202152436. PMID: 25642988.

\bibitem[3]{Goel}
Goel G, Daveson AJM, Hooi CE, Tye-Din JA, Wang S, Szymczak E, Williams LJ, Dzuris JL, Neff KM, Truitt KE, Anderson RP. Serum cytokines elevated during gluten-mediated cytokine release in coeliac disease. Clin Exp Immunol. 2020 Jan;199(1):68-78. doi: 10.1111/cei.13369. Epub 2019 Oct 1. PMID: 31505020; PMCID: PMC6904604.

\bibitem[4]{Hadjivassiliou}
Hadjivassiliou M, Gibson A, Davies-Jones GA, Lobo AJ, Stephenson TJ, Milford-Ward A. Does cryptic gluten sensitivity play a part in neurological illness? Lancet 1996;347:369–371

\bibitem[5]{Luostarinen}
Luostarinen L, Pirttila T, Collin P. Coeliac disease presenting with neurological disorders. Eur Neurol 1999;42:132–135.

\bibitem[6]{Pellecchia}
Pellecchia MT, Scala R, Filla A, De Michele G, Ciacci C, Barone P. Idiopathic cerebellar ataxia associated with celiac disease: lack of distinctive neurological features. J Neurol Neurosurg Psychiatry 1999;66:32–35.

\bibitem[7]{Burk}
Burk K, Bosch S, Muller CA, Melms A, Zuhlke C, Stern M, Besenthal I, Skalej M, Ruck P, Ferber S, Klockgether T, Dichgans J. Sporadic cerebellar ataxia associated with gluten sensitivity. Brain 2001;124:1013–1019.

\bibitem[8]{Bushara}
Bushara KO. Neurologic presentation of celiac disease. Gastroenterology. 2005 Apr;128(4 Suppl 1):S92-7. doi: 10.1053/j.gastro.2005.02.018. PMID: 15825133.

\bibitem[9]{Lebwohl}
Lebwohl B, Sanders DS, Green PHR. Coeliac disease. Lancet. 2018 Jan 6;391(10115):70-81. doi: 10.1016/S0140-6736(17)31796-8. Epub 2017 Jul 28. PMID: 28760445.

\bibitem[10]{Duarte}
Duarte Lobo D, Nobre RJ, Oliveira Miranda C, Pereira D, Castelhano J, Sereno J, Koeppen A, Castelo-Branco M, Pereira de Almeida L. The blood-brain barrier is disrupted in Machado-Joseph disease/spinocerebellar ataxia type 3: evidence from transgenic mice and human post-mortem samples. Acta Neuropathol Commun. 2020 Aug 31;8(1):152. doi: 10.1186/s40478-020-00955-0. PMID: 32867861; PMCID: PMC7457506.

\bibitem[11]{Hong}
Hong J, Yoon D, Nam Y, Seo D, Kim JH, Kim MS, Lee TY, Kim KS, Ko PW, Lee HW, Suk K, Kim SR. Lipopolysaccharide administration for a mouse model of cerebellar ataxia with neuroinflammation. Sci Rep. 2020 Aug 7;10(1):13337. doi: 10.1038/s41598-020-70390-7. PMID: 32770064; PMCID: PMC7414878.

\bibitem[12]{Cicarelli}
Cicarelli G, Della Rocca G, Amboni M, Ciacci C, Mazzacca G, Filla A, Barone P. Clinical and neurological abnormalities in adult celiac disease. Neurol Sci. 2003 Dec;24(5):311-7. doi: 10.1007/s10072-003-0181-4. PMID: 14716525.

\bibitem[13]{Ciacci}
Ciacci C, Iavarone A, Mazzacca G, De Rosa A. Depressive symptoms in adult coeliac disease. Scand J Gastroenterol. 1998 Mar;33(3):247-50. doi: 10.1080/00365529850170801. PMID: 9548616.

\bibitem[14]{Troubat}
Troubat R, Barone P, Leman S, Desmidt T, Cressant A, Atanasova B, Brizard B, El Hage W, Surget A, Belzung C, Camus V. Neuroinflammation and depression: A review. Eur J Neurosci. 2021 Jan;53(1):151-171. doi: 10.1111/ejn.14720. Epub 2020 Mar 20. PMID: 32150310.

\bibitem[15]{Kim}
Kim YK, Na KS, Myint AM, Leonard BE. The role of pro-inflammatory cytokines in neuroinflammation, neurogenesis and the neuroendocrine system in major depression. Prog Neuropsychopharmacol Biol Psychiatry. 2016 Jan 4;64:277-84. doi: 10.1016/j.pnpbp.2015.06.008. Epub 2015 Jun 23. PMID: 26111720.

\bibitem[16]{DosSantos}
DosSantos MF, Holanda-Afonso RC, Lima RL, DaSilva AF, Moura-Neto V. The role of the blood-brain barrier in the development and treatment of migraine and other pain disorders. Front Cell Neurosci. 2014 Oct 8;8:302. doi: 10.3389/fncel.2014.00302. PMID: 25339863; PMCID: PMC4189386.

\bibitem[17]{Mahmoudi}
Mahmoudi J, Mohaddes G, Erfani M, Sadigh-Eteghad S, Karimi P, Rajabi M, Reyhani-Rad S, Farajdokht F. Cerebrolysin attenuates hyperalgesia, photophobia, and neuroinflammation in a nitroglycerin-induced migraine model in rats. Brain Res Bull. 2018 Jun;140:197-204. doi: 10.1016/j.brainresbull.2018.05.008. Epub 2018 May 9. PMID: 29752991.

\bibitem[18]{Loscher}
Löscher W, Friedman A. Structural, Molecular, and Functional Alterations of the Blood-Brain Barrier during Epileptogenesis and Epilepsy: A Cause, Consequence, or Both? Int J Mol Sci. 2020 Jan 16;21(2):591. doi: 10.3390/ijms21020591. PMID: 31963328; PMCID: PMC7014122.

\bibitem[19]{Volta}
Volta U, De Giorgio R, Granito A, Stanghellini V, Barbara G, Avoni P, Liguori R, Petrolini N, Fiorini E, Montagna P, Corinaldesi R, Bianchi FB. Anti-ganglioside antibodies in coeliac disease with neurological disorders. Dig Liver Dis. 2006 Mar;38(3):183-7. doi: 10.1016/j.dld.2005.11.013. Epub 2006 Feb 7. PMID: 16458087.

\bibitem[20]{Briani}Briani C, Ruggero S, Zara G, Toffanin E, Ermani M, Betterle C, Guariso G. Anti-ganglioside antibodies in children with coeliac disease: correlation with gluten-free diet and neurological complications. Aliment Pharmacol Ther. 2004 Jul 15;20(2):231-5. doi: 10.1111/j.1365-2036.2004.02016.x. PMID: 15233704.

\bibitem[21]{Volta2}
Volta U, Molinaro N, De Franchis R, Forzenigo L, Landoni M, Fratangelo D, Bianchi FB. Correlation between IgA antiendomysial antibodies and subtotal villous atrophy in dermatitis herpetiformis. J Clin Gastroenterol. 1992 Jun;14(4):298-301. doi: 10.1097/00004836-199206000-00007. PMID: 1607605.

\bibitem[22]{Barrell}
Barrell K, Smith AG. Peripheral Neuropathy. Med Clin North Am. 2019 Mar;103(2):383-397. doi: 10.1016/j.mcna.2018.10.006. Epub 2018 Dec 17. PMID: 30704689.

\bibitem[23]{Friebe}
Friebe A., Horn M., Schmidt F., Janssen G., Schmid-Wendtner M.H., Volkenandt M., Hauschild A., Goldsmith C.H., Schaefer M. Dose-dependent development of depressive symptoms during adjuvant interferon-alpha treatment of patients with malignant melanoma. Psychosomatics. 2010;51(6):466–473. [PMID: 21051677].

\bibitem[24]{Eggermont}
Eggermont A.M., Suciu S., Santinami M., Testori A., Kruit W.H., Marsden J., Punt C.J., Salès F., Gore M., Mackie R., Kusic Z., Dummer R., Hauschild A., Musat E., Spatz A., Keilholz U. Adjuvant therapy with pegylated interferon alfa-2b versus observation alone in resected stage III melanoma: final results of EORTC 18991, a randomised phase III trial. Lancet. 2008;372(9633):117–126.

\bibitem[25]{Reichenberg}
Reichenberg A., Gorman J.M., Dieterich D.T. Interferon-induced depression and cognitive impairment in hepatitis C virus patients: a 72 week prospective study. AIDS. 2005;19(Suppl. 3):S174–S178.

\bibitem[26]{Kohler}
Kohler O, Krogh J, Mors O, Benros ME. Inflammation in Depression and the Potential for Anti-Inflammatory Treatment. Curr Neuropharmacol. 2016;14(7):732-42. doi: 10.2174/1570159x14666151208113700. PMID: 27640518; PMCID: PMC5050394.

\bibitem[27]{Kiecold-Glaser}
Kiecolt-Glaser JK, Derry HM, Fagundes CP. Inflammation: depression fans the flames and feasts on the heat. Am J Psychiatry. 2015 Nov 1;172(11):1075-91. doi: 10.1176/appi.ajp.2015.15020152. Epub 2015 Sep 11. PMID: 26357876; PMCID: PMC6511978.

\bibitem[28]{Edvinsson}
Edvinsson L, Haanes KA, Warfvinge K. Does inflammation have a role in migraine? Nat Rev Neurol. 2019 Aug;15(8):483-490. doi: 10.1038/s41582-019-0216-y. Epub 2019 Jul 1. PMID: 31263254.

\bibitem[29]{He}
He W, Long T, Pan Q, Zhang S, Zhang Y, Zhang D, Qin G, Chen L, Zhou J. Microglial NLRP3 inflammasome activation mediates IL-1β release and contributes to central sensitization in a recurrent nitroglycerin-induced migraine model. J Neuroinflammation. 2019 Apr 10;16(1):78. doi: 10.1186/s12974-019-1459-7. PMID: 30971286; PMCID: PMC6456991.

\bibitem[30]{Bjork}
Björck S, Lindehammer SR, Fex M, Agardh D. Serum cytokine pattern in young children with screening detected coeliac disease. Clin Exp Immunol. 2015 Feb;179(2):230-5. doi: 10.1111/cei.12454. PMID: 25212572; PMCID: PMC4298400.

\bibitem[31]{DePaolo}
DePaolo RW, Abadie V, Tang F, et al. Co-adjuvant effects of retinoic acid and IL-15 induce inflammatory immunity to dietary antigens. Nature. 2011;471(7337):220-224. doi:10.1038/nature09849

\bibitem[32]{Yokoyama}
Yokoyama S, Watanabe N, Sato N, Perera PY, Filkoski L, Tanaka T, Miyasaka M, Waldmann TA, Hiroi T, Perera LP. Antibody-mediated blockade of IL-15 reverses the autoimmune intestinal damage in transgenic mice that overexpress IL-15 in enterocytes. Proc Natl Acad Sci U S A. 2009 Sep 15;106(37):15849-54. doi: 10.1073/pnas.0908834106. Epub 2009 Sep 1. PMID: 19805228; PMCID: PMC2736142.

\bibitem[33]{Abadie}
Abadie, V., Kim, S.M., Lejeune, T. et al. IL-15, gluten and HLA-DQ8 drive tissue destruction in coeliac disease. Nature 578, 600–604 (2020). https://doi.org/10.1038/s41586-020-2003-8

\bibitem[34]{Obrenovich}
Obrenovich MEM. Leaky Gut, Leaky Brain? Microorganisms. 2018 Oct 18;6(4):107. doi: 10.3390/microorganisms6040107. PMID: 30340384; PMCID: PMC6313445.

\bibitem[35]{Binienda}
Binienda A, Twardowska A, Makaro A, Salaga M. Dietary Carbohydrates and Lipids in the Pathogenesis of Leaky Gut Syndrome: An Overview. Int J Mol Sci. 2020 Nov 8;21(21):8368. doi: 10.3390/ijms21218368. PMID: 33171587; PMCID: PMC7664638.

\bibitem[36]{Fändriks}
Fändriks L. Roles of the gut in the metabolic syndrome: an overview. J Intern Med. 2017 Apr;281(4):319-336. doi: 10.1111/joim.12584. Epub 2016 Dec 19. PMID: 27991713.

\bibitem[37]{Simeonova}
Simeonova D, Ivanovska M, Murdjeva M, Carvalho AF, Maes M. Recognizing the Leaky Gut as a Trans-diagnostic Target for Neuroimmune Disorders Using Clinical Chemistry and Molecular Immunology Assays. Curr Top Med Chem. 2018;18(19):1641-1655. doi: 10.2174/1568026618666181115100610. PMID: 30430944.

\bibitem[38]{Obrenovich2}
Obrenovich, M.; Rai, H.; Chittoor Mana, T.S.; Shola, D.; McCloskey, B.; Sass, C.; Levison, B. Dietary co-metabolism within the microbiota-gut-brain-endocrine metabolic interactome. BAO Microbiol. 2007, 2, 022.

\bibitem[39]{Reilly}
Reilly NR, Green PH. Epidemiology and clinical presentations of celiac disease. Semin Immunopathol. 2012 Jul;34(4):473-8. doi: 10.1007/s00281-012-0311-2. Epub 2012 Apr 24. PMID: 22526468.

\bibitem[40]{Janigro}
Janigro D. Are you in or out? Leukocyte, ion, and neurotransmitter permeability across the epileptic blood-brain barrier. Epilepsia. 2012 Jun;53 Suppl 1(0 1):26-34. doi: 10.1111/j.1528-1167.2012.03472.x. PMID: 22612806; PMCID: PMC4093790.

\bibitem[41]{Benros}
Benros ME, Waltoft BL, Nordentoft M, Ostergaard SD, Eaton WW, Krogh J, Mortensen PB. Autoimmune diseases and severe infections as risk factors for mood disorders: a nationwide study. JAMA Psychiatry. 2013 Aug;70(8):812-20. doi: 10.1001/jamapsychiatry.2013.1111. PMID: 23760347.

\bibitem[42]{Aflatoonian}
Aflatoonian M, Sivandzadeh G, Morovati-Sharifabad M, Mirjalili SR, Akbarian-Bafghi MJ, Neamatzadeh H. ASSOCIATIONS OF IL-6 -174G>C AND IL-10 -1082A>G POLYMORPHISMS WITH SUSCEPTIBILITY TO CELIAC DISEASE: EVIDENCE FROM A META-ANALYSIS AND LITERATURE REVIEW. Arq Gastroenterol. 2019 Sep 30;56(3):323-328. doi: 10.1590/S0004-2803.201900000-60. PMID: 31633733.

\bibitem[43]{Hourigan}
Hourigan CS. The molecular basis of coeliac disease. Clin Exp Med. 2006 Jun;6(2):53-9. doi: 10.1007/s10238-006-0095-6. PMID: 16820991.

\bibitem[44]{Chai}
Chai NC, Peterlin BL, Calhoun AH. Migraine and estrogen. Curr Opin Neurol. 2014 Jun;27(3):315-24. doi: 10.1097/WCO.0000000000000091. PMID: 24792340; PMCID: PMC4102139.

\bibitem[45]{Dodick}
Dodick DW. A Phase-by-Phase Review of Migraine Pathophysiology. Headache. 2018 May;58 Suppl 1:4-16. doi: 10.1111/head.13300. PMID: 29697154.

\bibitem[46]{Kemper}
R. Kemper, W. Meijler, J. Korf, G. Ter Horst
Migraine and function of the immune system: a meta-analysis of clinical literature published between 1966 and 1999
Cephalalgia, 21 (2001), pp. 549-557

\bibitem[47]{Braniste}
Braniste V, Al-Asmakh M, Kowal C, Anuar F, Abbaspour A, Tóth M, Korecka A, Bakocevic N, Ng LG, Kundu P, Gulyás B, Halldin C, Hultenby K, Nilsson H, Hebert H, Volpe BT, Diamond B, Pettersson S. The gut microbiota influences blood-brain barrier permeability in mice. Sci Transl Med. 2014 Nov 19;6(263):263ra158. doi: 10.1126/scitranslmed.3009759. Erratum in: Sci Transl Med. 2014 Dec 10;6(266):266er7. Guan, Ng Lai [corrected to Ng, Lai Guan]. PMID: 25411471; PMCID: PMC4396848.

\bibitem[48]{Braniste}
Braniste V, Al-Asmakh M, Kowal C, Anuar F, Abbaspour A, Tóth M, Korecka A, Bakocevic N, Ng LG, Kundu P, Gulyás B, Halldin C, Hultenby K, Nilsson H, Hebert H, Volpe BT, Diamond B, Pettersson S. The gut microbiota influences blood-brain barrier permeability in mice. Sci Transl Med. 2014 Nov 19;6(263):263ra158. doi: 10.1126/scitranslmed.3009759. Erratum in: Sci Transl Med. 2014 Dec 10;6(266):266er7. Guan, Ng Lai [corrected to Ng, Lai Guan]. PMID: 25411471; PMCID: PMC4396848.

\end{thebibliography}

\end{document}

